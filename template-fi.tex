% --- Template for thesis / report with tktltiki2 class ---
% 
% last updated 2013/02/15 for tkltiki2 v1.02

\documentclass[finnish]{tktltiki2}

% tktltiki2 automatically loads babel, so you can simply
% give the language parameter (e.g. finnish, swedish, english, british) as
% a parameter for the class: \documentclass[finnish]{tktltiki2}.
% The information on title and abstract is generated automatically depending on
% the language, see below if you need to change any of these manually.
% 
% Class options:
% - grading                 -- Print labels for grading information on the front page.
% - disablelastpagecounter  -- Disables the automatic generation of page number information
%                              in the abstract. See also \numberofpagesinformation{} command below.
%
% The class also respects the following options of article class:
%   10pt, 11pt, 12pt, final, draft, oneside, twoside,
%   openright, openany, onecolumn, twocolumn, leqno, fleqn
%
% The default font size is 11pt. The paper size used is A4, other sizes are not supported.
%
% rubber: module pdftex

% --- General packages ---

\usepackage[utf8]{inputenc}
\usepackage[T1]{fontenc}
\usepackage{lmodern}
\usepackage{microtype}
\usepackage{amsfonts,amsmath,amssymb,amsthm,booktabs,color,enumitem,graphicx}
\usepackage[pdftex,hidelinks]{hyperref}
\usepackage{setspace}
\doublespacing

% Automatically set the PDF metadata fields
\makeatletter
\AtBeginDocument{\hypersetup{pdftitle = {\@title}, pdfauthor = {\@author}}}
\makeatother

% --- Language-related settings ---
%
% these should be modified according to your language

% babelbib for non-english bibliography using bibtex
\usepackage[fixlanguage]{babelbib}
\selectbiblanguage{finnish}

% add bibliography to the table of contents
\usepackage[nottoc]{tocbibind}
% tocbibind renames the bibliography, use the following to change it back
\settocbibname{Lähteet}

% --- Theorem environment definitions ---

\newtheorem{lau}{Lause}
\newtheorem{lem}[lau]{Lemma}
\newtheorem{kor}[lau]{Korollaari}

\theoremstyle{definition}
\newtheorem{maar}[lau]{Määritelmä}
\newtheorem{ong}{Ongelma}
\newtheorem{alg}[lau]{Algoritmi}
\newtheorem{esim}[lau]{Esimerkki}

\theoremstyle{remark}
\newtheorem*{huom}{Huomautus}


% --- tktltiki2 options ---
%
% The following commands define the information used to generate title and
% abstract pages. The following entries should be always specified:

\title{Pariohjelmoinnin taloudelliset hyödyt}
\author{Ville Knuuttila}
\date{\today}
\level{Kandidaatintutkielma}
\abstract{Tutkielmassa tutustutaan pariohjelmoinnin taloudellisiin hyötyihin. Pariohjelmoinnilla tiedetään olevan positiivinen vaikutus ohjelmakoodin laatun, mutta vie keskimääräisesti enemmän henkilötyötunteja, kuin yksin ohjelmoidessa. Onko pariohjelmointi taloudellisesi varteenotettava ohjelmistokehitysmuoto? Selvitämme paraneeko ohjelmakoodin laatu niin paljon että ohjelman ylläpitovaiheessa saadaan koravattua ylimenneet henkilötyötunnit toteutusvaiheesta.}

% The following can be used to specify keywords and classification of the paper:

\keywords{pariohjelmointi, taloudellisuus}
\classification{} % classification according to ACM Computing Classification System (http://www.acm.org/about/class/)
                  % This is probably mostly relevant for computer scientists

% If the automatic page number counting is not working as desired in your case,
% uncomment the following to manually set the number of pages displayed in the abstract page:
%
% \numberofpagesinformation{16 sivua + 10 sivua liitteissä}
%
% If you are not a computer scientist, you will want to uncomment the following by hand and specify
% your department, faculty and subject by hand:
%
% \faculty{Matemaattis-luonnontieteellinen}
% \department{Tietojenkäsittelytieteen laitos}
% \subject{Tietojenkäsittelytiede}
%
% If you are not from the University of Helsinki, then you will most likely want to set these also:
%
% \university{Helsingin Yliopisto}
% \universitylong{HELSINGIN YLIOPISTO --- HELSINGFORS UNIVERSITET --- UNIVERSITY OF HELSINKI} % displayed on the top of the abstract page
% \city{Helsinki}
%


\begin{document}

% --- Front matter ---

\maketitle        % title page
\makeabstract     % abstract page

\tableofcontents  % table of contents
\newpage          % clear page after the table of contents


% --- Main matter ---

\section{Johdanto}

Pariohjelmointi on ohjelmointimenetelmä, jossa kaksi ohjelmoijaa istuvat saman koneen ääressä ohjelmoimassa~\cite{pairprogramming2}. Pariohjelmoinnissa ohjelmoijilla on kaksi eri roolia: kontrolloija ja tarkkailija. Kontrolloija on henkilö, joka kirjoittaa ohjelmakoodia eli käyttää näppäimistöä ja hiirtä. Tarkkailija istuu kontrolloijan vieressä nähden monitorin kokonaan ja etsii virheitä koodista. 


Pariohjelmoinnilla on todettu olevan ohjelmakoodin laatuun ja parien ongelmanratkaisukykyyn positiivisia vaikutuksia~\cite{pairprogramming}. Ongelmat ratkeavat jopa 60 prosentissa siitä ajasta, mitä yksin ohjelmoivat henkilöt käyttävät saman ongelman ratkaisuun~\cite{meta}. Virheiden määrä ohjelmissa myös pienenee jopa neljäsosan~\cite{williams00str}. Näiden lisäksi pariohjelmoidessa ohjelmoijalla on suurempi kynnys käydä tarkistamassa omaa henkilökohtaista sähköpostiaan tai käydä lukemassa aiheeseen liittymättömiä verkkosivuja~\cite{williams03pair}. Ohjelmoijat kuitenkin arvostavat toistensa aikaa, eivätkä halua tuhlata sitä omilla henkilökohtaisilla asioillaan. Näin ollen myös keskittyminen pysyy paremmin ongelmanratkaisussa tai tehtävän toteuttamisessa.

Taloudellisuus on nyky-yhteiskunnassa merkittävä tekijä. Jos pariohjelmointi voidaan todeta taloudellisesti kannattavaksi, voitaisiin se ottaa laajemmin käyttöön myös yritysmaailmassa. Tällöin voisimme tulevaisuudessa nauttia laadukkaammasta ja virheettömämmästä ohjelmakoodista.

Toisaalta jos pariohjelmointi ei ole taloudellisesti kannattavaa, niin voitaisiin sen käyttöä entisestään vähentää. Pariohjelmointia voitaisiin käyttää vain sillioin kun siitä on hyötyä muuten, esim. jos ongelma on todella haastava, tai yrityksellä on uusi työntekijä joka pitäisi totuttaa uuten ohjelmakoodikantaan.

Pariohjelmoijat kuitenkin käyttävät keskimäärin 120-150 \% enemmän henkilötyötunteja ongelman tai tehtävän ratkaisuun kuin yksin ohjelmoivat henkilöt~\cite{williams01support}. Tämä tarkoittaa sitä, että esimerkiksi kaupallisen ohjelman ohjelmointi tulee maksamaan työnantajalle ohjelman toteutusvaiheessa jopa 1,5 kertaa enemmän, jos ohjelma ohjelmoidaan käyttäen pariohjelmointia. Ohjelmankoodin hyvä laatu helpottaa ohjelman ylläpitoa ja jatkokehitystä~\cite{pearse95maintainability}. Näin ollen ylläpitovaiheeseen tarvittavat tunnit pienenevät. Tämän tutkielman tavoitteena on tarkastella pariohjelmoinnin taloudellista kannattavuutta henkilötyötunneissa mitattuna.

\section{Pariohjelmointi}

Pariohjelmoinnin määrittelyn mukaan pariohjelmointi on sitä, kun kaksi ohjelmoijaa työstävät samaa tehtävää tai ongelmaa yhdellä tietokoneella~\cite{nawrocki01exp}. Tässä kappaleessa käsittelemme pariohjelmoinnin historiaa, sen prosessia ja parivariaatioita ohjelmoijien kokemuksen perusteella, sekä pariohjelmoinnin hyötyjä ja haittoja.

\subsection{Pariohjelmoinnin historia}

Pariohjelmointia on harrastettu pidempään kuin sitä on edes kutsuttu pariohjelmoinniksi~\cite{williams03pair}. Varhaisimpia viittauksia pariohjelmointiin löytyy vuosilta 1953-1956~\cite{williams96pair}. 1980-luvulla tehdyissä tutkimuksissa havaittiin, että yritysmaailmassa ohjelmoijat käyttävät suurimman osan ajastaan tehden töitä muiden ihmisten kanssa~\cite{lister87peopleware}. Vain 20 prosenttia työajasta käytetään yksilöohjelmointiin, 50 prosenttia käytetään parin kanssa ja 30 prosenttia kahden tai useamman henkilön kanssa työn tekemiseen.


1990-luvun puolivälissä ohjelmistotuotannossa ruvettiin olemaan yhä enenevissä määrin kiinnostuneita ketteristä ohjelmistotuotantomenetelmistä~\cite{martin2003agile}. Pariohjelmointikin nousi sen seurauksena pinnalle, kun se listatiin yhtenä XP-ohjelmistokehityksen kahdestatoista käytänteestä~\cite{beck00extreme}.

\subsection{Pariohjelmoinnin prosessi}

Pariohjelmointi prosessina on hyvin yksinkertainen. Siinä toinen parista kirjoittaa ohjelmakoodia ja toinen katselmoi sitä jatkuvasti. Parien olisi syytä vaihtaa rooleja riittävän usein~\cite{beck00extreme}. 

Ohjelmointikokemuksen perusteella parityypit pystytään jakamaan kolmeen eri variaatioon: kokenut-kokenut, kokenut-kokematon ja kokematon-kokematon. Tämän lisäksi pariohjelmoinnille tyyppilisenä piirteenä on roolien, kontrolloija ja tarkkailija, vaihtaminen tietyin väliajoin~\cite{williams02support}.

Ohjelmoinnin opetukseen ja kodikantaan tutustumiseen hyödyllisin variaatio on, että kontroillajana toimii kokenut ohjelmoija~\cite{chong2007social}. Tällöin pelkästään kirjoittamalla ohjelmakoodia kontrolloija kykenee opettamaan kokemattomammalle tarkkailijalle hyviä käytänteitä ja koodikannan sisältöä. Erityisen tärkeää on kiinnittää huomiota siihen, että vuorovaikutus pysyy molempi suuntaisena. Tarkkailijan pitää kyetä kertomaan omat mielipiteensä kokeneemalle kontrolloijalle.

Nopeimman ongelmaratkaisukyvyn saavuttaa, jos sekä kontrolloija että tarkkailija ovat molemmat kokeneita koodaajia~\cite{voas2001faster}. Jos ongelma on haastava, ohjelmakoodin laatu paranee entisestään,  mutta vaatii henkisesti suuremman ponnistuksen. Todella yksinkertaisiin tehtäviin voi kuitenkin mennä enemmän aikaa kuin yksin ohjelmoidessa.

Uusille ohjelmoijille hyvä tapa opettaa luettavan koodin kirjoittamista, on pistää sekä tarkkailijaksi että kontrolloijaksi aloitteleva ohjelmoija~\cite{chong2007social}. Tämä pakottaa kontrolloijan kirjoittamaan ohjelmakoodinsa sellaiseksi että tarkkailija saa siitä selvää. Huonona puolena on, että ongelmanratkaisukyky ei kauheasti nopeudu verrattuna yksinohjelmoivaan aloittelijaan.

\begin{center}
    \begin{tabular}{ | p{4,5cm} | p{5cm} | p{4cm} | }
    \hline
   	Parivariaatio & Hyödyt & Haitat \\ \hline
    	Kokenut - Kokenut & Nopea ongelmanratkaisu & Yksinkertaiset ongelmat saattavat viedä huomattavasti enemmän aikaa\\ \hline
	Kokenut - Kokematon & Opetustarkoitus ja uuteen ohjelmakoodikantaan tutustuttaminen & Vuorovaikutus vaikeaa, ja siihen pitää kiinnittää paljon huomiota\\ \hline
	Kokematon - Kokematon & Opettaa kirjoittamaan luettavaa koodia & Ongelman ratkaisu kestää lähes yhtä kauan kuin yksin ohjelmoivalla \\ \hline
    \end{tabular}
\end{center}


\subsection{Pariohjelmoinnin hyödyt ja haitat}

Pariohjelmoidessa kaksi ihmistä istuvat saman tietokoneen ääressä tekemässä samaa tehtävää, tai toisin sanoen kaksi ihmistä tekevät yhden ihmisen työt. Tämä herättää kysymyksen, mitkä ovat pariohjelmoinnin hyödyt ja haitat todellisuudessa. Pariohjelmoinnilla on pakko olla hyötyjä  ~\cite{costandbenefit}, tai sitä ei muuten harjoitettaisi, eikä Kent Beck olisi varmasti nimennyt sitä XP-ohjelmistokehityksen yhdeksi käytänteeksi.

Mitkä ovat sitten pariohjelmoinnin tunnetuimmat haitat. Yleensä ottaen johtotason henkilöt näkevät ohjelmoijat harvana resurssina ~\cite{costandbenefit2} eivätkä halua hukata näitä resursseja keskittämällä useampaa ohjelmoijaa tekemään samaa osaa ohjelmakoodista. Ohjelmointia ollaan myös yleensä opetettu yksinäisen aktiviteettina, josta johtuu monen kokeneen ohjelmoijan haluttomuus ohjelmoida muiden ihmisten kanssa. Osa jopa pitää omaa ohjelmakoodiaan niin persoonallisena, että toinen ihminen vain hidastaisi heitä. Tämän lisäksi yksinkertaisiin tehtäviin saattaa mennä melkein yhtäkauan kuin yhdellä ohjelmoijallakin menisi.

Ongelmanratkaisunopeus on yksi tunnetuimmista pariohjelmoinnin hyödyistä ~\cite{costandbenefit2}. Mitä kokeneempia ohjelmoijia ja haastavampia ongelmia, sitä suuremman hyödyn pariohjelmoinnista voi saada. Tämän lisäksi pariohjelmointia käyttäen tuotettu ratkaisu on ohjelmakoodin rivimäärältä yleensä huomattavasti pienempi, jopa $\frac{4}{5}$ siitä mitä yksinohjelmoidessa tehtynä. Ohjelmakoodin rivimäärän vähäisyys johtaa myös siihen että laatu on usein parempaa kuin yksin ohjelmoidessa. Näin ollen pariohjelmoinnilla tuotettujen ohjelmien jatkokehitys ja ylläpito on helpompaa.


\section{Taloudellisuus ohjelmistotuotannossa}

Ohjelmistotuotanto koostuu kahdesta eri vaiheesta: toteutuksesta ja ylläpidosta ~\cite{sommerville1998requirements}. Ohjelmiston kehitys alkaa aina toteutusviheella, joka koostuu neljästä vaiheesta: suunnittelu, toteutus, testaus ja käyttöönotto. Ylläpitovaihe seuraa aina toteutusta. Kun ohjelma on otettu käyttöön, niin siirrytään ylläpitovaiheeseen. Ylläpito voi olla täysin olematonta, mutta silti käyttöönoton jälkeistä aikaa kutsutaan ylläpitovaiheeksi. Toteutusvaiheeseen voidaan palata ylläpitovaiheesta. Termi jota tästä käytetään on jatkokehitys. Jatkokehitys ei sinäänsä eroa mitenkään muuten toteutuksesta, kuin että ohjelmalle ollaan vähintään kerran suoritettu jo käyttöönotto ja siirrytty ylläpitovaiheeseen.

Ohjelmistotuotannossa suurin osa taloudellisista menoista tulee henkilöstöstä~\cite{haikala1995ohjelmistotuotanto}. Tietokoneiden ja muiden tarvittavien laitteiden osuus kuluista on huomattavasti pienempi kuin mitä työntekijöiden vaatimat kustannukset. Suomessa henkilöstökustannukset yleensä koostuvat työntekijöille maksettavasta palkasta, vakuutuksista ja eläkemaksuista. Osa kustannuksista tulee myös työntekijöiden viihdyttämisestä, koska on todettu, että tyytyväinen työntekijä on yritykselle arvokkaampi kuin tyytmätön~\cite{airo2008oma}.

Suurimpia kustannuksia voidaan siis mitata laskemalla aikaa, joka ohjelmiston toteutukseen ja ylläpitoon menee. Kuinka nopeasti suunnittelu, toteutus, testaus ja käyttöönotto saadaan toteutettua ja kuinka helppoa ylläpito on.  Mitä vähemmän virheitä ohjelman toteutusvaiheessa ohjelmakoodiin jää, sen helpompaa ja vähäisempää ohjelman ylläpito on, eli vie vähemmän aikaa työntekijöiltä.

Luettavampi ja parempilaatuisempi ohjelmakoodi johtaa siihen, että ohjelman jatkokehittäminen on helpompaa ~\cite{johnson1994instrumented}. Näin ollen, jos ohjelmaa jatkokehitetään niin tulee halvemmaksi, jos alkuperäisen ohjelman ohjelmakoodi on kirjoitettu laadukkaammaksi ~\cite{fagan2001design}. Koodikatselmointi on tapa jolla koodin laatua pystytään parantamaan ja valvomaan. Katselmoidessa joku toinen ohjelmointitaitoinen henkilö lukee kirjoitetun ohjelmakoodin läpi ja merkitsee ohjelmakoodiin virheet tai huonot käytänteet, jotka varsinainen ohjelmoija sen jälkeen korjaa. 


\section{Kysymys}

Pariohjelmoinnissa toteutusvaiheessa käytetään enemmän henkilötyötunteja kuin yksin ohjelmoidessa~\cite{costandbenefit2}. Tässä tutkielmassa tarkastelen, paraneeko ohjelmakoodin laatu niin paljon, että ylläpitovaiheessa tarvittavilla henkilötyötunneilla saadaan yhteensä tarvittavat henkilötyötunnit pienemmäksi pariohjelmoidessa kuin yksilöinä ohjelmoidessa. Tarkastelu pohjautuu jo olemassa oleviin tutkimuksiin.\newline\newline
$X_{total} = Yksilöohjelmoinnin henkilötyötunnit $\newline
$X_{impl} = Yksilöohjelmoinnin toteutusvaiheen henkilötyötunnit $\newline
$X_{maint} = Yksilöohjelmoinnin ylläpitovaiheen henkilötyötunnit $\newline
$Y_{total} = Pariohjelmoinnin henkilötyötunnit $\newline
$Y_{impl} = Pariohjelmmoinnin toteutusvaiheen henkilötyötunnit $\newline
$Y_{maint} = Pariohjelmoinnin ylläpitovaiheen henkilötyötunnit $\newline
$X_{total} = X_{impl} + X_{maint} $\newline
$Y_{total} = Y_{impl} + Y_{maint} $\newline
$\Rightarrow Y_{total} < X_{total}$ \newline

Ohjelmistotuotannon suurimmat kustannukset tulevat henkilöstökuluista~\cite{haikala1995ohjelmistotuotanto}, joten jonkun tietyn ohjelmistotuotantomenetelmän taloudellinen kannattavuus on tietyllä tavalla mahdollista laskea tarkastelemalla sen vaativia henkilöstö resursseja. Näistä henkilötyötunnit ovat helpoiten mitattavia yksiköitä. Näin ollen tarkastelemalla: onko pariohjelmoinnin henkilötyötunnit pienemmät kuin yksilöohjelmoinnin henkilötyötunnit, saadaan kuva siitä onko pariohjelmointi taloudellisesti kannattavaa.


\section{Case}
\begin{center}
    \begin{tabular}{ | l | l | l | l | l | p{4cm} |}\hline
    Tutkimus & Impl mh & Main mh & Koodikanta & context & tavat \\ \hline
    	~\cite{case} & x \% & y \% & 50k & organization wide & pro pro vaikea ongelma \\ \hline
	~\cite{costandbenefit} & 1,4 \% & 0,7 \% & 0 & - & keskimäärin \\ \hline
    \end{tabular}
\end{center}

\section{Tulokset}
Oliko tot2 < tot1

\newpage

% --- Back matter ---
%
% bibtex is used to generate the bibliography. The babplain style
% will generate numeric references (e.g. [1]) appropriate for theoretical
% computer science. If you need alphanumeric references (e.g [Tur90]), use
%
% \bibliographystyle{babalpha-lf}
%
% instead.

\bibliographystyle{babalpha-lf}
\bibliography{references-fi}


\end{document}
